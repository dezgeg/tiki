\section{Johdanto}

Ohjelmistojen haavoittuvuukset ovat verrattaen ikäviä:
Tietoturva-aukoista aiheutuneesta ylimääräisestä työstä tietojärjestelmien ylläpitäjille
sekä menetetystä työajasta voi seurata suuria rahallisia tappioita,
puhumattakaan mahdollisista henkilötietojen tai yrityssalaisuuksien vuotamisesta.
Esimerkiksi Microsoftin IIS-palvelimen haavoittuvuuden avulla levinneestä Code Red-madosta
arvioidaan aiheutuneen yhteensä noin 2,6 miljardin tappiot~\cite{CodeRed}.
Siksi on toivottavaa, että ohjelmiston tietoturvasta voidaan varmistua ennen ohjelmiston käyttöönottoa.

Tietoturvaongelmia voi yrittää löytää käsityönä tutkimalla ohjelman lähdekoodia,
mikä tietenkin on mahdollista vain ohjelmiston varsinaisille kehittäjille
tai avoimen lähdekoodin ohjelmille.
Lähdekoodin puuttuessa täytyy ensin ohjelmatiedosto takaisinkääntää
disassembler-ohjelmalla symboliselle konekielelle
ja tutkia ohjelmaa konekielitasolla.
Kummassakin tapauksessa manuaalinen tutkiminen on työlästä ja aikaavievää, joten automatisoitu ratkaisu on paikallaan.

Tarkastellaan aluksi kappaleessa \ref{YleinenTietoturva} C-kielisissä ohjelmissa yleisiä tietoturvaongelmia,
muistivirheitä, sekä kappaleesta \ref{Testausmenetelmat} eteenpäin yleisiä tietoturvatestaamisen menetelmiä.

\section{Yleisiä tietoturvaongelmia}
\label{YleinenTietoturva}

Tietoturvahaavoittuvuuden vakavuuteen vaikuttaa keskeisesti aiheutuuko siitä \emph{luottamusrajan}
(trust boundary) ylitys~\cite{ViolatingAssumptions}.
Esimerkiksi jos jonkun palvelun kaatavan ohjelmointivirheen laukaiseminen vaatii käyttäjältä pääsyä
ylläpitäjän hallintavalikkoon, ei kyseisellä virheellä ole suhteellisesti kovin suurta tietoturvamerkitystä.
Jos käyttäjällä kerran on ylläpitäjän oikeudet, niin samaan lopputulokseenhan päädytään mikäli kyseinen
käyttäjä yksinkertaisesti sammuttaa palvelimen tavallisia keinoja käyttäen.
Sen sijaan jos vaikkapa Javascript-sovelmat selaimessa pääsee lukemaan käyttäjän tiedostoja,
on tilanne vakavampi,
koska web-sivujen katselemisen tarkoitus on olla turvallista ja selaimen siten
kuuluu estää sovelmien pääsy tiedostojärjestelmään.
Vastaavaan tapaan tavallisen käyttäjän pääsy suorittamaan ohjelmia ylläpitäjän oikeuksilla
tai käyttöjärjestelmätilassa rikkoo luottamusrajan~\cite{ViolatingAssumptions}.

\subsection{Muistiturvallisuus}

Ohjelmointikieli vaikuttaa ratkaisevasti siihen miten laajoja seurauksia ohjelmointivirheillä voi olla tietoturvan kannalta.
Suurin ohjelmointikielestä riippuva tekijä on kielen \emph{muistiturvallisuus}~\cite{SoftBound},
eli tunnistaako ajonaikainen ympäristö kiellettyjen muistiviitteiden tekemisen.
Nykypäivänä yleisimmät muistiturvallisuuden puutteesta kärsivät ohjelmointikielet ovat C ja C++,
joita tehokkuussyistä edelleen käytetään laajasti~\cite{StaticallyDetecting,SoftBound}.

Muistivirheet ilmenevät useimmiten prosessin keskeytymisenä tietyllä
käyt\-tö\-jär\-jes\-tel\-mä\-koh\-tai\-sel\-la virhekoodilla.
Unix-pohjaisissa järjestelmissä tämä virhe on tunnettu `Segmentation fault'.
% ja Windows-järjestelmissä STATUS\_ACCESS\_VIOLATION.
Koska muistivirheistä aiheutuu useimmiten ohjelman kaatuminen,
mahdollistaa muistivirheen olemassaolo \emph{palvelunestohyökkäyksen} (DoS-hyökkäys eli Denial-of-Service -hyökkäys).
Kriittisimmillään muistivirheet voivat mahdollistaa hyökkääjän suorittaa mielivaltaista koodia
ohjelmaprosessin käyttöoikeuksilla (RCE, Remote Code Execution).
Kuitenkin pelkällä prosessin pakotetulla sulkeutumisellakin voi olla muitakin lisäseurauksia.
Esimerkiksi prosessilla ei ole mahdollisuutta siivota jotakin säilyvää tilaa,
kuten väliaikaistiedostoja.
Hallitsematon kuolema voi myös aiheuttaa tiedon korruptoitumista jos ohjelma kaatuu esimerkiksi kesken levykirjoitusta.

Muistivirheiksi luokitellaan tyypillisesti seuraavat:

\begin{itemize}
    \item Taulukon yli tai ali indeksointi: C-kielen taulukoissa ohjelmoijan vastuulla on huolehtia,
          ettei taulukkoa indeksoida liian suurella tai negatiivisella indeksillä.
          Jos näin pääsee tapahtumaan, kielen spesifikaatio ei ota kantaa siihen mitä tapahtuu~\cite[6.5.6]{CSpec}.
% An array subscript is out of range, even if an object is apparently accessible with the
% given subscript (...) (6.5.6).
          Käytännössä usein tapahtuu joko muistiviittaus taulukkoa ympäröiviin muuttujiin tai
          ajonaikaisympäristön tietorakenteisiin.
          Suurin riski pinossa olevan taulukon ohi kirjoittamisessa onkin aktivaatiotietueessa
          sijaitsevan funktion paluuosoitteen ylikirjoittaminen~\cite{StaticallyDetecting,SplintLCLint}.
          Tästä seuraa käytännössä suoraan mielivaltaisen koodin suoritus.
    \item Puskurin ylivuoto: Vastaavasti kuin taulukoiden kanssa,
          C-kielessä täytyy merkkijonoja ja tavupuskureita kopioitaessa tai luettaessa
          ohjelmoijan itse huolehtia, että puskurissa on riittävästi tilaa~\cite[7.24.1]{CSpec}.
% If an array is accessed beyond the end of an object, the behavior is undefined.
          Seuraukset ovat enimmäkseen samat kuin taulukon yli indeksoimisella.
          Puskureiden ylivuoto on C:ssä harmillisen helppo tehdä.
          Jo C:n standardikirjastosta löytyy funktioita joille ei ole ollenkaan mahdollista antaa tulospuskurin
          kokoa ja siten aiheuttavat ylivuodon jos tulos ei mahdu puskuriin~\cite{StaticallyDetecting,SplintLCLint}.
          Tällaisia funktioita ei koskaan pitäisi käyttää syötteen käsittelyn yhteydessä.
          Puskurin ylivuodot pinossa ovatkin yleinen syy RCE-haavoittuvuuksille~\cite{SplintLCLint}.
    \item Alustamattoman muistin käyttö: C:ssä keosta tai paikallisille muuttujille varattua muistia
          alusteta mitenkään.
          Kielen spesifikaatio ei määrittele mitä alustamattoman muistin käytöstä seuraa~\cite[6.3.2.1 ja 7.22.3.4]{CSpec},
          mutta yleensä alustamattoman muistialueen sisältönä on jotain mitä kyseisen muistialueen aiempi käyttäjä on
          sinne sattunut kirjoittamaan~\cite{SecurityRootOfTheProblem}.
% An lvalue designating an object of automatic storage duration that could have been
% declared with the register storage class is used in a context that requires the value
% of the designated object, but the object is uninitialized. (6.3.2.1).
%
% The malloc function allocates space for an object whose size is specified by size and
% whose value is indeterminate. (7.22.3.4)
          Jos tällaista muistia sitten näytetään jossain muodossa käyttäjälle,
          saattaa käyttäjälle vuotaa esimerkiksi ohjelman käyttämiä salausavaimia,
          ohjelman muille käyttäjille kuuluvaa tietoa tai
          riittävästi tietoa ohjelman käyttämistä muistialueista helpottamaan RCE-haavoittuvuuden tekoa.
          Vaihtoehtoisesti alustamattoman osoitinmuuttujan käyttö johtaa hyvin todennäköisesti uusiin muistivirheisiin
          ja ohjelman kaatumiseen.

\end{itemize}
Edellisten lisäksi C-ohjelmoijan täytyy huoleehtia dynaamisesti varatun muistin vapauttamisesta.
Tästä aiheutuu vielä lisää mahdollisia virhetilanteita:

\begin{itemize}
    \item Muistivuodot: Jos ohjelmoija unohtaa vapauttaa dynaamisesti varatun muistialueen
          sen jälkeen kun sitä ei enää käytetä,
          jää tuo muistivaraus kuluttamaan muistia ohjelman sulkemiseen asti.
          Tämä mahdollistaa esimerkiksi palvelunestohyökkäyksen~\cite{SplintLCLint},
          mikäli lisääntynyt muistinkäyttö johtaa ohjelman sivutukseen levylle.
      \item Muistin vapauttaminen useasti (\emph{double free}):
          Varattua dynaamista muistialuetta ei saa vapauttaa kuin yhden kerran.
          Muistialueen vapauttamisen jälkeen sama muistialue voidaan antaa esimerkiksi
          ohjelman jonkun toisen moduulin muistivarauksen käyttöön.
          Jos nyt alkuperäistä, jo vapautettua muistialuetta yritetään vapauttaa,
          mikä näinollen aiheuttaisi jonkun toisen, täysin liittymättömän muistialueen vapauttamisen.
          Yleensä vapautetun muistialueen vapauttaminen uudelleen aiheuttaa muistinhallintakirjaston
          sisäisten tietorakenteiden korruptoitumisen,
          joka voi tietyissä tilanteissa johtaa ulkopuolisen koodin suoritukseen~\cite{DoubleFree}.
\end{itemize}

Muistivirheiden vakavuuden ja yleisyyden takia lähes kaikki tietoturvatestaamisen menetelmät
kykenevät löytämään muistivirheitä.
Perehdytäänkin seuraavassa kappaleessa näihin testausmenetelmiin.

\section{Testausmenetelmät}
\label{Testausmenetelmat}

Ohjelmistotekniikan menetelmistä tuttu laadunvarmistustekniikka on automaattiset testit~\cite{Sommerville}.
Testausta voidaankin soveltaa tietoturvaongelmien välttämiseen tietyin edellytyksin:
sen sijaan, että testataan toivotun toiminnallisuuden olemassaoloa,
testataankin \emph{epätoivotun käytöksen} puutetta~\cite{OuluBrowser}.
Yleisempiä epätoivottuja tapahtumia ovat kaatumiset ja jumiutumiset (esimerkiksi ikisilmukat).

\subsection{Luokittelu}

Testauskeinoja voidaan tavallisten funktionaalisten testien tapaan luokitella karkeasti
\emph{blackbox}- ja \emph{whitebox}-testeiksi sen mukaan kuinka paljon
testaaminen kohdistuu ohjelmiston tavanomaisiin rajapintoihin ja kuinka paljon
ohjelmiston sisäisen rakenteen toimintaan~\cite{Sommerville}.

Blackbox-testauksessa ohjelmatiedostoa ajetaan aivan normaalisti,
ja tutkitaan sen käyttäytymistä ohjelmaan sopivilla syötteillä~\cite{OuluBrowser}.
Esimerkiksi palvelinohjelmiston ollessa kyseessä siihen avataan verkkoyhteys,
tai komentoriviohjelmalle annetaan syöte normaalisti komentoriviparametrien kautta.
Whitebox-testauskeinoissa kajotaan ohjelman sisäiseen rakenteeseen.
Tä\-män\-kal\-tai\-seen testaukseen on tapoja huomattavasti enemmän.
Esimerkiksi jotkut keinot voivat vaatia pääsyä ohjelman lähdekoodiin,
tai sitten ohjelman suoritusta voidaan analysoida tai muuttaa konekielitasolla~\cite{OuluBrowser}.

\subsection{Testauksen kriteereitä}

Tarkastellaan seuraavaksi muutamaa automaattista testausmenetelmää: \emph{staattista analyysiä}
kappaleessa \ref{StaattinenAnalyysi} ja \emph{fuzzausta} kappaleessa \ref{Fuzzaus}.

\section{Staattinen analyysi}
\label{StaattinenAnalyysi}

Staattiseen analyysiin perustuvat keinot ovat lähdekoodin analyysiä tekeviä whitebox-menetelmiä.
Tällaiset menetelmät yrittävät paikantaa lähdekoodista tyypillisiä ohjelmointivirheitä~\cite[22.3]{Sommerville}.
Monista tällaisista ongelmista saattaa olla seuraamuksia tietoturvan kannalta~\cite{StaticCodeAnalysis}.
Yleisin tällainen keino on kääntäjien tarjoamat varoitukset ohjelmiston käännösaikana.
Usein käytettyjä staattisen analyysin työkaluja ohjelmointivirheiden etsintään
ovat \emph{lint}-tyyliset~\cite{Lint} ohjelmistot eri ohjelmointikielille sekä
esimerkiksi Coverity~\cite{Coverity}.

\subsection{Lint}

Lint~\cite{Lint} on varhainen C-kielisiä ohjelmia tarkistava staattisen analyysin työkalu vuodelta 1978.
Lintin fokuksena eivät olleet varsinaisesti tietoturvaongelmat,
vaan yksinkertaiset ohjelmointivirheet, C-kääntäjää tarkempi tyypintarkastus sekä siirrettävyysongelmat.
Lint osaa varoittaa esimerkiksi seuraavista virheistä:~\cite{Lint}

\begin{itemize}
    \item Alustamattoman muuttujan arvon käyttö.
    \item Käyttämättömät muuttujat tai funktiot.
    \item Saavuttamattomat koodirivit.
    \item ``Järjettömät'' vertailut.
           Esimerkiksi etumerkittömälle kokonaislukumuuttujalle \texttt{x} ei ehtolauseke
           \texttt{if (x < 0)} ole koskaan tosi.
   \item Yllättävä operaattoripresedenssi.
         Esimerkiksi \texttt{if (x \& 0x10 == 0)} näyttää päällisin puolin bittitarkistukselta,
         mutta C:ssä \texttt{==}\,-operaattori sitoo \texttt{\&}-operaattoria vahvemmin,
         jollon ehtolausekkeen tulokseksi tulee aina \texttt{0}.
\end{itemize}

Tuon ajan C-kääntäjissä kääntämisen nopeus oli korkeammalla prioriteetilla,
joten tämänkaltaisetkin kriittiset tarkastukset ulkoistettiin Lint-ohjelmalle~\cite{Lint}.
Esimerkiksi C-kääntäjät tyypillisesti käsittelevät yhtä C-tiedostoa kerrallaan,
kun taas Lint tarkastelee tiedostoja kokonaisuutena,
mikä mahdollistaa tarkemman analyysin~\cite{Lint}.
Nykyiset C-kääntäjät osaavat varoittaa e\-del\-lä\-mai\-ni\-tuis\-ta virheistä
jo käännösaikana~\cite{SecurityRootOfTheProblem} ja uudempien ohjelmointikielten,
esimerkiksi Javan, spesifikaatiot vaativat tällaiset tarkistukset~\cite[22.3]{Sommerville}.

Monet vastaavanlaiset staattisen analyysin työkalut eri kielille ovat nimetty
Lint-ohjelman mukaan, esimerkiksi JSLint JavaScript-kielelle.
Erityisesti C-kielen tietoturvaongelmiin keskittynyt staattisen analyysin ohjelmisto on Splint
(Secure Programming Lint), joka on aiemmin kulkenut nimellä LCLint.

Splint on keskittynyt löytämään samankaltaisia virheitä kuin Lint,
mutta lisäksi myös muistivirheitä sekä muistivuotoja C-kielisistä ohjelmista.
Muistivirheiden löytäminen tapahtuu esittämällä ohjelmakoodi lausekalkyylin toteutuvuusongelmana
(\emph{satisfiability problem}).
Ohjelmaan on määritelty jokaista C-kirjaston puskurinkäsittelyfunktiota kohti joukko
\emph{alkuehtoja}, joiden täytyy päteä funktiota kutsuttaessa.
Esimerkiksi kopioitaessa merkkijonoa \texttt{strcpy}-funktiolla ei osoitin kohde- tai lähdepuskuriin
saa olla \texttt{NULL} sekä kohdepuskurin koko täytyy olla vähintään yhtä suuri kuin lähdepuskurin.
Muistia varaavat funktiot ja konstruktiot sen sijaan tuottavat \emph{jälkiehtoja} -
esimerkiksi onnistuneen \texttt{malloc}-funktiokutsun tuloksena on puskuri, jonka
koko annettiin parametrinä.
Näiden symbolisten ehtojen perusteella Splint yrittää selvittää syntyykö
puskurin turvallisen käytön edellyttämistä ehdoista ja
puskurin todellisen koon aiheuttamista ehdoista ristiriita.
Tällöin puskurin ylivuoto on mahdollinen ja Splint antaa siitä varoituksen.

\section{Fuzzaus}
\label{Fuzzaus}

\emph{Fuzzaus} on yleisesti käytetty keino automaattiseen testisyötteiden generointiin.
Sen ideana on raa'asti luoda suuria määriä mahdollisia syötteitä satunnaislukujen pohjalta siinä toivossa,
että ohjelma ei käsittele niitä oikein~\cite{UnixReliability}.
Muun muassa aiemmin mainittuja muistivirheitä löytyy usein fuzzauksella:
normaalia reilusti pidemmät merkkijonot voivat ylivuotaa kiinteäkokoisia puskureita.
Samaten erikoisia merkkejä sisältävät merkkijonot sekä erittäin suuret tai negatiiviset lukuarvot
voivat olla ongelmallisia~\cite{ViolatingAssumptions},
aiheuttaen esimerkiksi aritmeettisiä ylivuotoja laskutoimituksissa.

Fuzzaukselle hyvin sopivia kohteita ovat muun muassa erinäisten tiedostoformaattien jäsentäjät~\cite{SageArtikkeli,OuluBrowser}:
esimerkiksi web-selaimet joutuvat käsittelemään muun muassa
HTML-, CSS-, PNG- ja JavaScript-muotoisia tiedostoja~\cite{OuluBrowser}.
Tämä on varsin suuri määrä koodia,
joka joutuu käsittelemään syötteenään tiedostoja suoraan lähtökohtaisesti epäturvallisesta Internetistä,
jolloin hyökkäyspinta-alaakin on runsaasti~\cite{OuluBrowser}.

Fuzzaukseen sopivien syötteiden luontiin on olemassa runsaasti tekniikoita
aina yksinkertaisista blackbox-menetelmistä monimutkaisempiin keinoihin:
\begin{itemize}
    \item Yksinkertaisimmillaan syöte voi olla lähes pelkkää satunnaista dataa.
          Esimerkiksi monien Unixin komentorivityökalujen syöte koostuu pelkistä tekstiriveistä
          ilman sen kummempaa rakennetta,
          jolloin jono rivinvaihdoilla eroteltua satunnaisgeneroituja
          tavuja~\cite{UnixReliability} on riittävä syöte ohjelmalle.
    \item Ennalta olemassa olevia kelvollisia syötteitä voidaan \emph{mutatoida} esimerkiksi
          lisäämällä, poistamalla tai muokkaamalla sen osia.
    \item Rakenteellisesti enimmäkseen kelvollisia, mutta normaalisti harvoin esiintyviä syötteitä voidaan
          luoda esimerkiksi kontekstittoman kieliopin tai jonkun muun formaalin syötteen määrittelyn perusteella.
    \item Ohjelman suoritusta voidaan analysoida symbolisesti välttämään syötteitä,
          jolla ei ole vaikutusta ohjelmaan~\cite{SageArtikkeli}.
\end{itemize}

Käsitellään seuraavaksi joitakin edellämainittuja keinoja generoida syötteitä.

\subsection{Testitapausten generointi}

Fuzzauksen historia alkaa kirjaimellisesti synkkänä ja myrskyisenä yönä,
jolloin sään aiheuttamat tiedonsiirtovirheet modeemiyhteydessä koituivat ongelmaksi
eräälle alkuperäisen fuzzauspaperin kirjoittajalle.
Tiedonsiirtovirheiden aiheuttamat satunnaiset merkit linjalle aiheutti nimittäin
useiden Unixin peruskomentorivityökalujen kaatumisen~\cite{UnixReliability}.
Tämän innoittamana kehitettiin \emph{fuzz}-niminen työkalu Unix-komentoriviohjelmien testaamiseen.

Fuzz-ohjelma yksinkertaisesti tulosti standarditulostusvirtaansa satunnaisia tavuja,
jotka sitten yhdistettiin Unix-putkella testattavan ohjelman standardisyötevirtaan~\cite{UnixReliability}.
Fuzzin komentoriviparametreillä pystyi hieman vaikuttamaan fuzzin generoimaan syötteeseen:
syötteen pituuden pystyi määrittää sekä ASCII-kontrollimerkit sai kytkettyä päälle tai pois.
Kokonaisuudessaan fuzzin generoimassa syötteessä ei ollut lähes ollenkaan rakennetta;
tämä kuitenkin riitti kaatamaan useat Unix-ohjelmat.

Testattavalle ohjelmalle annettu syöte voi ohjelmasta riippuen olla
hyvinkin erityyppistä kuin edellä esitelty tavujonoista koostuva syöte.
Esimerkiksi graafisen käyttöliittymän toteutus toimii usein tapahtumapohjaisesti:
sekä Unixin X11:n, Mac OS X:n Aquan sekä Microsoft Windowsin ikkunanhallinta on toteutettu näin.
Näihin graafisiin järjestelmiin perustuvia ohjelmia voi siten fuzzata lähettämällä
niille esimerkiksi tekaistuja näppäimistön painalluksia tai hiiren liikkeitä vastaavia
sanomia~\cite{X11Fuzz,MacOsFuzz,WinNtFuzz}.

Esimerkki rakenteellisen syötteen generoinnista on \emph{regfuzz}-ohjelma,
joka on kehitelty säännöllisten lausekkeiden (regular expression) moottoreiden
fuzzaamiseen~\cite{RegFuzz}.
Regfuzz lähtee rakentamaan säännöllistä lauseketta pala kerrallaan sisältä ulospäin:
liikkeelle lähdetään tyhjästä merkkijonosta ja siihen lisätään joko täysin
satunnaisia merkkejä tai valideja rakenteita,
kuten muun muassa sulkeita, merkkijoukkoja (esimerkiksi \texttt{[\^{}a-z]}) ja
operaattoreita (kuten \texttt{+}, \texttt{*}).
Tätä prosessia toistetaan niin kauan kunnes on saatu halutun kokoinen säännöllinen lauseke.

\section{Yhteenveto}
Tässä tutkielmassa tarkasteltiin kahta menetelmää ohjelmistojen tietoturvatestaamiseen,
staattista analyysia ja fuzzausta.
Huomattavaa on, että koska osa tietoturvaongelmista on seurausta ohjelmointivirheistä,
niin tietoturvan parantamiseen soveltuu ohjelmointivirheitä yleisesti paikantavat ohjelmistotekniikan menetelmät.
Kuitenkin fuzzaus testausmenetelmänä on toimiva menetelmä erityisesti tietoturvaongelmien löytämiseen.
