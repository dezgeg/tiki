\section{Johdanto}

Ohjelmistojen haavoittuvuuksilla voi olla ikäviä seurauksia.
Siksi on toivottavaa, että mahdolliset tietoturva-aukot löydetään mielellään jo ennen tuotantoon asentamista.
Haavoittuvuuksia voi löytää muun muassa manuaalisesti auditoimalla ohjelmakoodi.
Tämä kuitenkin on työlästä ja aikaavievää, joten automatisoituja testausmenetelmiä suositaan enemmän.
Perinteisten ohjelmistotekniikan testausmenetelmien lisäksi tietoturvatestaukseen on kehitelty uudenlaisia testausmenetelmiä.


Testauskeinoja luokitellaan karkeasti \emph{white box}- ja \emph{black box}-testeiksi.
Black box-testauksessa ohjelmistoa testataan antamalla sille syötteitä tavanomaiseen tapaan ilman että sen sisäistä toimintaa tarkastellaan.
Tyypillisesti tämänkaltaisessa testauksessa havainnoidaan ohjelmasta ainoastaan selkeätä ei-toivottua käytöstä, kuten kaatumista tai jumiutumista.
