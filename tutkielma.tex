% --- Template for thesis / report with tktltiki2 class ---

\documentclass[finnish]{tktltiki2/tktltiki2}

% --- General packages ---

\usepackage[utf8]{inputenc}
\usepackage{lmodern}
\usepackage{microtype}
\usepackage{amsfonts,amsmath,amssymb,amsthm,booktabs,color,enumitem,graphicx,listings,float}
\usepackage[pdftex,hidelinks]{hyperref}

% Fuck you latex.
\lstset{
  literate={ö}{{\"o}}1
           {ä}{{\"a}}1
}

% Automatically set the PDF metadata fields
\makeatletter
\AtBeginDocument{\hypersetup{pdftitle = {\@title}, pdfauthor = {\@author}}}
\makeatother

% --- Language-related settings ---
%
% these should be modified according to your language

% babelbib for non-english bibliography using bibtex
\usepackage[fixlanguage]{babelbib}
\selectbiblanguage{finnish}

% add bibliography to the table of contents
\usepackage[nottoc]{tocbibind}
% tocbibind renames the bibliography, use the following to change it back
\settocbibname{Lähteet}

% --- Theorem environment definitions ---

\newtheorem{lau}{Lause}
\newtheorem{lem}[lau]{Lemma}
\newtheorem{kor}[lau]{Korollaari}

\theoremstyle{definition}
\newtheorem{maar}[lau]{Määritelmä}
\newtheorem{ong}{Ongelma}
\newtheorem{alg}[lau]{Algoritmi}
\newtheorem{esim}[lau]{Esimerkki}

\theoremstyle{remark}
\newtheorem*{huom}{Huomautus}


% --- tktltiki2 options ---
%
% The following commands define the information used to generate title and
% abstract pages. The following entries should be always specified:

\title{Ohjelmistojen haavoittuvuustestaus}
\author{Tuomas Tynkkynen}
\date{\today}
\level{Kandidaatintutkielma}
\abstract{
Ohjelmistojen tietoturvahaavoittuvuuksista voi aiheutua ikäviä seurauksia yrityksen maineen ja ihmisten yksityisyyden kannalta.
On siis sekä kehittäjien että käyttäjien edun mukaista,
että ohjelmiston tietoturvasta voidaan varmistua ennen sen käyttöönottoa.
Tässä tutkielmassa perehdytään aluksi ohjelmistojen yleisiin haavoittuvuuksiin sekä
esitellään automaattisia menetelmiä ohjelmistojen haavoittuvuustestaukseen.
Erityisesti keskitytään C-kielisten ohjelmien muistivirheisiin liittyviin haavoittuvuuksiin.

Testausmenetelmistä käsitellään staattista analyysiä, dynaamista instrumentointia sekä fuzzausta.
Vaikka kyseiset menetelmät osoittautuvat toimivan varsin eri tavalla,
soveltuvat ne kaikki muistivirheiden paikantamiseen.
}

% The following can be used to specify keywords and classification of the paper:

\keywords{automaattinen testaus, ohjelmistojen tietoturva}
\classification{Security and privacy $\rightarrow$ \textbf{Vulnerability management}\\
Security and privacy $\rightarrow$ \emph{Software security engineering}}

% If the automatic page number counting is not working as desired in your case,
% uncomment the following to manually set the number of pages displayed in the abstract page:
%
% \numberofpagesinformation{16 sivua + 10 sivua liitteissä}
%
% If you are not a computer scientist, you will want to uncomment the following by hand and specify
% your department, faculty and subject by hand:
%
% \faculty{Matemaattis-luonnontieteellinen}
% \department{Tietojenkäsittelytieteen laitos}
% \subject{Tietojenkäsittelytiede}
%
% If you are not from the University of Helsinki, then you will most likely want to set these also:
%
% \university{Helsingin Yliopisto}
% \universitylong{HELSINGIN YLIOPISTO --- HELSINGFORS UNIVERSITET --- UNIVERSITY OF HELSINKI} % displayed on the top of the abstract page
% \city{Helsinki}
%

\linespread{1.5}
\newcommand{\fixme}[1][FIXME!]{\textcolor{red}{*}\marginpar{\colorbox{red}{\scriptsize{#1}}}}
% \newcommand{\fixme}[1][FIXME!]{}

\begin{document}

% --- Front matter ---


\frontmatter      % roman page numbering for front matter
\maketitle        % title page
\makeabstract     % abstract page

\tableofcontents  % table of contents

% --- Main matter ---

\mainmatter
\section{Johdanto}

Ohjelmistojen haavoittuvuukset ovat verrattaen ikäviä:
tietoturva-aukoista aiheutuneesta ylimääräisestä työstä tietojärjestelmien ylläpitäjille
sekä menetetystä työajasta voi seurata suuria rahallisia tappioita,
puhumattakaan mahdollisista henkilötietojen tai yrityssalaisuuksien vuotamisesta.
Esimerkiksi Microsoftin IIS-palvelimen haavoittuvuuden avulla levinneestä Code Red-madosta
arvioidaan aiheutuneen yhteensä noin 2,6 miljardin tappiot~\cite{CodeRed}.
Siksi on toivottavaa, että ohjelmiston tietoturvasta voidaan varmistua ennen ohjelmiston käyttöönottoa.

Tietoturvaongelmia voi yrittää löytää käsityönä tutkimalla ohjelman lähdekoodia,
mikä tietenkin on mahdollista vain ohjelmiston varsinaisille kehittäjille
tai avoimen lähdekoodin ohjelmille.
Lähdekoodin puuttuessa täytyy ensin ohjelmatiedosto takaisinkääntää
disassembler-ohjelmalla symboliselle konekielelle
ja tutkia ohjelmaa konekielitasolla.
Kummassakin tapauksessa ihmisvoimin tapahtuva tutkiminen on työlästä ja aikaavievää, joten automatisoitu ratkaisu on paikallaan.

Tarkastelen tässä tutkielmassa erityisesti C- ja C++-kielisten ohjelmien tyypillisiä tietoturvahaavoittuvuuksia
ja automaattisia menetelmiä niiden löytämiseen.
Kappaleessa \ref{YleinenTietoturva} käsittelen tietoturvaongelmia sekä yleisellä tasolla
että erityisesti C-kielisiä ohjelmia vaivaaviin muistivirheisiin.
Kappaleessa \ref{Testausmenetelmat} kerron yleisesti tietoturvatestaamisen menetelmistä
sekä kappaleissa \ref{StaattinenAnalyysi}, \ref{DynaaminenInstrumentointi} ja \ref{Fuzzaus}
tarkemmin kolmesta menetelmästä: staattisesta analyysista, dynaamisesta instrumentoinnista sekä fuzzauksesta.

\section{Tietoturvahaavoittuvuudet}
\label{YleinenTietoturva}
\subsection{Yleisesti}
Tietoturvahaavoittuvuuden vakavuuteen vaikuttaa keskeisesti aiheutuuko siitä \emph{luottamusrajan}
(trust boundary) ylitys~\cite{ViolatingAssumptions}.
Esimerkiksi jos jonkun palvelun kaatavan ohjelmointivirheen laukaiseminen vaatii käyttäjältä pääsyä
ylläpitäjän hallintavalikkoon, ei kyseisellä virheellä ole suhteellisesti kovin suurta tietoturvamerkitystä.
Jos käyttäjällä kerran on ylläpitäjän oikeudet, niin samaan lopputulokseen päädytään mikäli kyseinen
käyttäjä yksinkertaisesti sammuttaa palvelimen tavallisia keinoja käyttäen.
Sen sijaan jos vaikkapa Javascript-sovelmat selaimessa pääsevät lukemaan käyttäjän tiedostoja,
on tilanne vakavampi,
koska web-sivujen katselemisen on tarkoitus olla turvallista ja selaimen siten
kuuluu estää sovelmien pääsy tiedostojärjestelmään.
Vastaavaan tapaan tavallisen käyttäjän pääsy suorittamaan ohjelmia ylläpitäjän oikeuksilla
tai käyttöjärjestelmätilassa rikkoo luottamusrajan~\cite{ViolatingAssumptions}.

Tietoturvahaavoittuvuuden vakavuuden luokitteluun on kehitelty joitakin suuntaa-antavia kvantitatiivisiä menetelmiä.
Näistä käytetyimpiä on NIST:n kehittämä ja standardisoitu CVSS\footnote{\url{http://nvd.nist.gov/cvss.cfm}} (Common Vulnerability Scoring System)~\cite{CVSS}.
Siinä yksittäisen haavoittuvuuden vakavuus määritellään lukuna välillä 0--10,
missä korkeampi luku tarkoittaa vakavampaa haavoittuvuutta.
Tämä pistemäärä muodostuu kuudesta eri osa-alueesta,
kuten kuinka monimutkaista haavoittuvuuden hyödyntäminen on ja kuinka paljon käyttöoikeuksia siihen vaaditaan.

CVSS:n ohella NIST ylläpitää tietokantaa kaikista levityksessä olevien ohjelmien julkiseksi tuoduista haavoittuvuuksista.
CVE-tietokanta\footnote{\url{http://cve.mitre.org}} on alan standardi ja julkaistuihin haavoittuuksiin viitataan usein `CVE-2013-1234'-muotoisella tunnisteella.
Tällaiset haavoittuvuustietokannat ovat mahdollistaneet haavoittuvuuksien analysoimisen ja tarkemman luokittelun.
Tietoturvahaavoittuvuuksia luokittelemalla on todettu,
että tietoturvaongelmiin johtavat ohjelmointivirheet ovat yhteisiä kaikille ohjelmistoille,
riippumatta niiden toimialasta.
Tällaiset yleispätevät haavoittuvuudet on kirjattu CWE-nimiseen tietokantaan (Common Weakness Enumeration).
Automaattisen tietoturvatestauksen kannalta tämä on erittäin hyvä asia,
sillä menetelmiä voidaan uudelleenkäyttää ohjelmasta toiseen pienemmällä vaivalla.

\begin{table}
\begin{tabular}{l|l|l}
    CVE-tunniste  & CVSS-pisteet    & Kuvaus (TODO: käännä) \\ \hline
%
    CVE-2013-0893 & 6.8 (Keskitaso) & Race condition in media handling.                       \\
    CVE-2013-0894 & 7.5 (Korkea)    & Buffer overflow in vorbis decoding.                     \\
    CVE-2013-0895 & 7.5 (Korkea)    & Incorrect path handling in file copying.                \\
    CVE-2013-0896 & 7.5 (Korkea)    & Memory management issues in plug-in message handling.   \\
    CVE-2013-0897 & 4.3 (Keskitaso) & Off-by-one read in PDF.                                 \\
    CVE-2013-0898 & 7.5 (Korkea)    & Use-after-free in URL handling.                         \\
    CVE-2013-0899 & 5.0 (Keskitaso) & Integer overflow in Opus handling.                      \\
\end{tabular}
\caption{Osa Chromium-selaimen 23.2.2013 julkistetuista haavoittuvuuksista (yhteensä 22).}
\label{chromiumTaulukko}
\end{table}

\subsection{Muistiturvallisuus}
\label{Muistivirheet}

Ohjelmointikieli vaikuttaa ratkaisevasti siihen miten laajoja seurauksia ohjelmointivirheillä voi olla tietoturvan kannalta.
Suurin ohjelmointikielestä riippuva tekijä on kielen \emph{muistiturvallisuus}~\cite{SoftBound},
eli tunnistaako ajonaikainen ympäristö kiellettyjen muistiviitteiden tekemisen.
Nykypäivänä yleisimmät muistiturvallisuuden puutteesta kärsivät ohjelmointikielet ovat C ja C++,
joita tehokkuussyistä edelleen käytetään laajasti~\cite{StaticallyDetecting,SoftBound}.

Muistivirheet ilmenevät useimmiten prosessin keskeytymisenä tietyllä
käyt\-tö\-jär\-jes\-tel\-mä\-koh\-tai\-sel\-la virhekoodilla.
Unix-pohjaisissa järjestelmissä tämä tunnetaan segmentointivirheenä (Segmentation fault).
% ja Windows-järjestelmissä STATUS\_ACCESS\_VIOLATION.
Koska muistivirheistä aiheutuu useimmiten ohjelman kaatuminen,
mahdollistaa muistivirheen olemassaolo \emph{palvelunestohyökkäyksen} (DoS-hyökkäys eli Denial-of-Service -hyökkäys).
Kriittisimmillään muistivirheet voivat sallia hyökkääjän suorittaa mielivaltaista koodia
ohjelmaprosessin käyttöoikeuksilla (RCE, Remote Code Execution).
Kuitenkin pelkällä prosessin pakotetulla sulkeutumisellakin voi olla lisäseurauksia.
Esimerkiksi prosessilla ei ole yleensä mahdollisuutta siivota väliaikais- tai lukkotiedostoja,
jotka voivat jäädä viemään levytilaa tai estämään ohjelman käyn\-nis\-tä\-mi\-sen,
mikäli ohjelma käyttää lukkotiedostoa es\-tä\-mään useamman ohjelmainstanssin ajamisen yhtäaikaa.
Hallitsematon sulkeutuminen voi myös aiheuttaa tiedon korruptoitumista jos ohjelma kaatuu esimerkiksi kesken levykirjoitusta.

Muistivirheiksi luokitellaan tyypillisesti muun muassa
taulukon yli tai ali indeksointi, puskurin ylivuoto sekä alustamattoman muistin käyttö:

\begin{itemize}
    \item Taulukon yli tai ali indeksointi: C-kielen taulukoissa ohjelmoijan vastuulla on huolehtia,
          ettei taulukkoa indeksoida liian suurella tai negatiivisella indeksillä.
          Jos näin pääsee tapahtumaan, kielen spesifikaatio ei ota kantaa siihen mitä tapahtuu~\cite[\S 6.5.6]{CSpec}.
% An array subscript is out of range, even if an object is apparently accessible with the
% given subscript (...) (6.5.6).
          Käytännössä usein tapahtuu joko muistiviittaus taulukkoa ympäröiviin muuttujiin tai
          ajonaikaisympäristön tietorakenteisiin.
          Suurin riski pinossa olevan taulukon ohi kirjoittamisessa onkin aktivaatiotietueessa
          sijaitsevan funktion paluuosoitteen ylikirjoittaminen~\cite{StaticallyDetecting,SplintLCLint}.
          Tästä seuraa käytännössä suoraan mielivaltaisen koodin suoritus.
    \item Puskurin ylivuoto: Vastaavasti kuin taulukoiden kanssa, \fixme[tämä on edellisen erikoistapaus]
          C-kielessä täytyy merkkijonoja ja tavupuskureita kopioitaessa tai luettaessa
          ohjelmoijan itse huolehtia, että puskurissa on riittävästi tilaa~\cite[\S 7.24.1]{CSpec}.
% If an array is accessed beyond the end of an object, the behavior is undefined.
          Seuraukset ovat enimmäkseen samat kuin taulukon yli indeksoimisella.
          Puskureiden ylivuoto on C:ssä harmillisen helppo aiheuttaa.
          Jo C:n standardikirjastosta löytyy funktioita joille ei ole lainkaan mahdollista antaa tulospuskurin
          kokoa ja siten aiheuttavat ylivuodon jos tulos ei mahdu puskuriin~\cite{StaticallyDetecting,SplintLCLint}.
          Tällaisia funktioita ei koskaan pitäisi käyttää syötteen käsittelyn yhteydessä.
          Puskurin ylivuodot pinossa ovatkin yleinen syy RCE-haavoittuvuuksille~\cite{SplintLCLint}.
    \item Alustamattoman muistin käyttö: C:ssä keosta tai paikallisille muuttujille
          varattua muistia ei alusteta mitenkään.
          Kielen spesifikaatio ei määrittele mitä alustamattoman muistin käytöstä seuraa~\cite[\S 6.3.2.1, \S 7.22.3.4]{CSpec},
          mutta yleensä alustamattoman muistialueen sisältönä on jotain mitä kyseisen muistialueen aiempi käyttäjä on
          sinne sattunut kirjoittamaan~\cite{SecurityRootOfTheProblem}.
% An lvalue designating an object of automatic storage duration that could have been
% declared with the register storage class is used in a context that requires the value
% of the designated object, but the object is uninitialized. (6.3.2.1).
%
% The malloc function allocates space for an object whose size is specified by size and
% whose value is indeterminate. (7.22.3.4)
          Jos tällaista muistia sitten näytetään jossain muodossa käyttäjälle,
          saattaa käyttäjälle vuotaa esimerkiksi ohjelman käyttämiä salausavaimia,
          ohjelman muille käyttäjille kuuluvaa tietoa tai
          riittävästi tietoa ohjelman käyttämistä muistialueista helpottamaan RCE-haavoittuvuuden hyödyntämistä.
          Vaihtoehtoisesti alustamattoman osoitinmuuttujan käyttö johtaa hyvin todennäköisesti uusiin muistivirheisiin
          ja ohjelman kaatumiseen.

\end{itemize}
Edellisten lisäksi C-ohjelmoijan täytyy huoleehtia dynaamisesti varatun muistin vapauttamisesta.
Tästä aiheutuu vielä lisää mahdollisia virhetilanteita,
kuten muistivuotoja tai dynaamisen muistin vapauttamista kahdesti:

\begin{itemize}
    \item Muistivuodot: Jos ohjelmoija unohtaa vapauttaa dynaamisesti varatun muistialueen
          sen jälkeen kun sitä ei enää käytetä,
          jää kyseinen muistivaraus kuluttamaan muistia ohjelman sulkemiseen asti.
          Tämä mahdollistaa esimerkiksi palvelunestohyökkäyksen~\cite{SplintLCLint},
          mikäli lisääntynyt muistinkäyttö johtaa ohjelman sivutukseen levylle.
      \item Muistin vapauttaminen moneen kertaan (\emph{double free}):
          Varattua dynaamista muistialuetta ei saa vapauttaa kuin yhden kerran.
          Muistialueen vapauttamisen jälkeen sama muistialue voidaan antaa esimerkiksi
          ohjelman jonkun toisen moduulin muistivarauksen käyttöön.
          Jos nyt alkuperäistä, jo vapautettua muistialuetta yritetään vapauttaa,
          mikä näinollen aiheuttaisi jonkun toisen, täysin liittymättömän muistialueen vapauttamisen.
          Yleensä vapautetun muistialueen vapauttaminen uudelleen aiheuttaa muistinhallintakirjaston
          sisäisten tietorakenteiden korruptoitumisen,
          joka voi tietyissä tilanteissa johtaa ulkopuolisen koodin suoritukseen~\cite{DoubleFree}.
\end{itemize}

Taulukossa \ref{chromiumTaulukko} on listattu joitakin viimeaikaisia Chromium-selaimen haavoittuvuuksia,
joista enemmistö on muistivirheitä,
mikä kertoo muistivirheiden yleisyydestä.
Muistivirheiden vakavuuden ja yleisyyden takia lähes kaikki tietoturvatestaamisen menetelmät
kykenevät löytämään muistivirheitä.
Tarkastellaan seuraavassa kappaleessa tietoturvatestaamisen menetelmiä.

\section{Testausmenetelmät}
\label{Testausmenetelmat}

Ohjelmistotekniikan menetelmistä tuttu laadunvarmistustekniikka on automaattiset testit~\cite[23.1]{Sommerville}.
Testausta voidaan soveltaa myös tietoturvaongelmien välttämiseen tietyin edellytyksin:
sen sijaan, että testataan toivotun toiminnallisuuden olemassaoloa,
testataankin \emph{epätoivotun käytöksen} puutetta~\cite{OuluBrowser}.
Yleisempiä epätoivottuja tapahtumia ovat kaatumiset ja jumiutumiset (esimerkiksi ikisilmukat).

% \subsection{Luokittelu}

Testauskeinoja voidaan tavallisten funktionaalisten testien tapaan luokitella karkeasti
\emph{blackbox}- ja \emph{whitebox}-testeiksi sen mukaan kuinka paljon
testaaminen kohdistuu ohjelmiston tavanomaisiin rajapintoihin ja kuinka paljon
ohjelmiston sisäisen rakenteen toimintaan~\cite{OuluBrowser}.

Blackbox-testauksessa ohjelmatiedostoa ajetaan aivan normaalisti,
ja tutkitaan sen käyttäytymistä ohjelmaan sopivilla syötteillä~\cite{OuluBrowser}.
Esimerkiksi palvelinohjelmiston ollessa kyseessä siihen avataan verkkoyhteys,
tai komentoriviohjelmalle annetaan syöte normaalisti komentoriviparametrien kautta.
Whitebox-testauskeinoissa kajotaan ohjelman sisäiseen rakenteeseen.
Tä\-män\-kal\-tai\-seen testaukseen on tapoja huomattavasti enemmän.
Esimerkiksi jotkut keinot voivat vaatia pääsyä ohjelman lähdekoodiin,
tai sitten ohjelman suoritusta voidaan analysoida tai muuttaa konekielitasolla~\cite{OuluBrowser}.

Tarkastellaan seuraavaksi kahta automaattista testausmenetelmää: staattista analyysiä
kappaleessa \ref{StaattinenAnalyysi} ja fuzzausta kappaleessa \ref{Fuzzaus}.

\section{Ohjelman ajonaikainen instrumentointi}
Luvussa \ref{Muistivirheet} tarkasteltiin muistivirheitä ja huomattiin,
että useat muistivirheet johtuvat ajonaikaisten tarkistusten puutteesta.
Jotta voitaisiin varmistua ohjelman olevan vapaa muistivirheistä,
olisi testauksen ajaksi suotavaa saada tällaiset ajonaikaiset tarkistukset päälle.
Tätä varten on kehitelty \emph{ohjelman ajonaikaiseen instrumentointiin} perustuvia työkaluja,
jolla käännettyyn ohjelmatiedostoon voidaan ajonaikana lisätä ylimääräisiä tarkistuksia.
Eräs tällainen työkalu on laajasti käytetty Valgrind.

\subsection{Valgrind}

Valgrind on Unix-pohjaisissa käyttöjärjestelmissä toimiva ohjelmien ajonaikaiseen instrumentointiin
perustuva työkalu, jota käytetään useimmiten muistivirheiden etsintään~\cite{Valgrind}.
Valgrind itsessään on laajempi kehys ajonaikaiseen instrumentointiin,
josta muistivirheitä löytävä \emph{Memcheck}-työkalu on ylivoimaisesti käytetyin.
Mikä tahansa käännetty ohjelma voidaan ajaa Valgrindin Memcheckin instrumentoimana ajamalla
\texttt{valgrind \emph{ohjelma}}.

Valgrind toimii muuntamalla suorituksessa olevan natiivikonekielilohkon omalle
arkkitehtuuririippumattomalle välikielelleen,
ja tekee sille tarvittavia työkalukohtaisia muunnoksia.
Memcheck-työkaluun instrumentointiin liittyy kaksi osa-aluetta:
sekä alustamattoman muistin että varaamattomien muistialueiden käytön seuranta.
Jälkimmäistä varten jokaista osoiteavaruuden tavua varten on varattu nk. A-bitti (accessible),
joka kertoo onko kyseinen tavu laillisesti ohjelman käytettävissä.
Ohjelman alussa ainoastaan itse muistissa oleva ohjelmatiedosto on merkattu käytettäväksi.
Memcheck kaappaa kutsut C:n muistinhallintafunktioihin,
ja joko asettaa tai poistaa A-bittejä kun keosta varataan tai vapautetaan muistia.
Lisäksi Memcheck instrumentoi ohjelman jokaisen muistiin viittaavan konekäskyn
ensiksi tarkistamaan osoitetta vastaavan A-bitin arvon,
ja aiheuttaa virheen mikäli ohjelma yrittää viitata varaamattomaan muistiin.
Näin voidaan konekäskyn tarkkuudella paikantaa missä kohtaa koodia
muistivirhe tapahtuu.

Alustamattoman muistin käytön seuranta on Memcheckille haastavampi operaatio.
Yksinkertainen ratkaisu, jossa tulostetaan virhe aina ladattaessa
tavu alustamattomasta muistista, ei nimittäin toimi hyvin, vaan aiheuttaa
useita aiheettomia varoituksia.
Tämä johtuu täytetavuista, joita C-kääntäjä saattaa joutua lisäämään struktuureiden keskelle.
Useissa prosessoreissa esimerkiksi neljätavuisen muuttujan muistiosoitteen täytyy olla neljällä jaollinen,
jolloin 1-tavuisen ja 4-tavuisen kentän sisältävään struktuuriin täytyy lisätä kolme täytetavua
1-tavuisen kentän jälkeen.
Kääntäjä ei koskaan käytä näitä tietorakenteiden täytetavuja mihinkään, eikä myöskään alusta niitä.
Kuitenkin kokonaisia struktuureita kopioivat funktiot, kuten \texttt{memcpy},
kopioivat nämä täytetavut muiden struktuurin kenttien mukana.
Tämä voi tapahtua täysin korrekteissa ohjelmissa,
joten virheilmoituksen antaminen tästä ei ole suotavaa.

Memcheckin käyttämä algoritmi on huomattavasti monimutkaisempi.
Erityisesti alustamattomia arvoja saa lukea muistista ja jopa käyttää useimmissa laskutoimituksissa ilman virheilmoitusta.
Memcheck ei anna varoitusta laskutoimituksen määrittelemättömästä arvosta niin kauan kun voidaan nähdä,
että alustamaton arvo ei ole \emph{havaittavasti} vaikuttanut ohjelman suoritukseen.
Tämä on toteutettu varaamalla jokaiselle ohjelman käsittelemälle bitille dataa (sekä rekistereissä että muistissa)
on varattu ylimääräinen V-bitti (valid),
joka kertoo onko kyseisen bitin arvo hyvinmääritelty vai ei.
Määrittelemättömiä arvoja syntyy `itsestään' vain kahdesta lähteestä:
funktiokutsun yhteydessä syntyvistä paikallisista muuttujista sekä varatusta dynaamisesta muistista.
Jälkimmäistä tapausta varten Memcheck yksinkertaisesti asettaa dynaamista muistia varattaessa uuden muistialueen kaikki V-bitit nollaksi.
Paikallisia muuttujia varten Memcheckin täytyy instrumentoida pinosta muistia varaavat käskyt (pino-osoitinta lisäävät tai vähentävät käskyt)
nollaamaan kyseisen alueen V-bitit.
Lisäksi jokainen dataa käsittelevä konekäsky instrumentoidaan päivittämään samalla lopputuloksen V-bittejä tarpeen mukaan:
esimerkiksi muistista rekisteriin lataava konekäsky asettaa myös kohderekisterin V-bitit muistista ladatun arvon V-biteistä.
Useiden konekäskyjen kohdalla yksikin määrittelemätön bitti aiheuttaa koko lopputuloksen olevan määrittelemätön.
Memcheck katsoo määrittelemättömien arvojen olevan ulkoisesti havaittavia kolmessa tapauksessa:
määrittelemätön arvo annetaan parametrina käyttöjärjestelmäkutsulle,
määrittelemätöntä arvoa käytetään muistiosoitteena,
tai ohjelman suoritusvuo riippuu määrittelemättömästä arvosta.

\begin{figure}
\lstinputlisting[language=C,numbers=left]{include/valgrindExample.c}
\lstinputlisting{include/valgrindOutput.txt}

\caption{Esimerkki Valgrind-ohjelman käytöstä.}
\end{figure}

\section{Lähdekoodin staattinen analyysi}
\label{StaattinenAnalyysi}

Staattinen analyysi on whitebox-menetelmä jossa yritetään paikantaa ohjelman
lähdekoodista tyypillisiä ohjelmointivirheitä~\cite[22.3]{Sommerville}.
Monista täl\-lai\-sis\-ta ongelmista saattaa olla seuraamuksia tietoturvan kannalta~\cite{StaticCodeAnalysis}.
Yleisin tällainen keino on kääntäjien tarjoamat varoitukset ohjelmiston käännösaikana.
Usein käytettyjä staattisen analyysin työkaluja ohjelmointivirheiden etsintään
ovat \emph{lint}-tyyliset~\cite{Lint} ohjelmistot eri ohjelmointikielille sekä
esimerkiksi Coverity~\cite{Coverity}.

% \subsection{Lint}

Lint~\cite{Lint} on varhainen C-kielisiä ohjelmia tarkistava staattisen analyysin työkalu vuodelta 1978.
Lintin fokuksena eivät olleet varsinaisesti tietoturvaongelmat,
vaan yksinkertaiset ohjelmointivirheet, C-kääntäjää tarkempi tyypintarkastus sekä siirrettävyysongelmat.
Lint osaa varoittaa esimerkiksi seuraavista virheistä~\cite{Lint}:

\begin{itemize}
    \item Alustamattoman muuttujan arvon käyttö.
    \item Käyttämättömät muuttujat tai funktiot.
    \item Saavuttamattomat koodirivit.
    \item ``Järjettömät'' vertailut.
           Esimerkiksi etumerkittömälle kokonaislukumuuttujalle \texttt{x} ei ehtolauseke
           \texttt{if (x < 0)} ole koskaan tosi.
   \item Yllättävä operaattoripresedenssi.
         Esimerkiksi \texttt{if (x \& 0x10 == 0)} näyttää päällisin puolin bittitarkistukselta,
         mutta C:ssä \texttt{==}\,-operaattori sitoo \texttt{\&}-operaattoria vahvemmin,
         jollon ehtolausekkeen tulokseksi tulee aina \texttt{0}.
\end{itemize}

Tuon ajan C-kääntäjissä kääntämisen nopeus oli korkeammalla prioriteetilla,
joten tämänkaltaisetkin kriittiset tarkastukset ulkoistettiin Lint-ohjelmalle~\cite{Lint}.
Esimerkiksi C-kääntäjät tyypillisesti käsittelevät yhtä C-tiedostoa kerrallaan,
kun taas Lint tarkastelee tiedostoja kokonaisuutena,
mikä mahdollistaa tarkemman analyysin~\cite{Lint}.
Nykyiset C-kääntäjät osaavat varoittaa e\-del\-lä\-mai\-ni\-tuis\-ta virheistä
jo käännösaikana~\cite{SecurityRootOfTheProblem} ja uudempien ohjelmointikielten,
esimerkiksi Javan, spesifikaatiot vaativat tällaiset tarkistukset~\cite[22.3]{Sommerville}.

Monet vastaavanlaiset staattisen analyysin työkalut eri kielille ovat nimetty
Lint-ohjelman mukaan, esimerkiksi JSLint JavaScript-kielelle.
Erityisesti C-kielen tietoturvaongelmiin keskittynyt staattisen analyysin ohjelmisto on Splint
(Secure Programming Lint), joka on aiemmin kulkenut nimellä LCLint~\cite{SplintLCLint}.

Splint on keskittynyt löytämään samankaltaisia virheitä kuin Lint,
mutta lisäksi myös muistivirheitä sekä muistivuotoja C-kielisistä ohjelmista.
Muistivirheiden löytäminen tapahtuu esittämällä ohjelmakoodi lausekalkyylin toteutuvuusongelmana
(\emph{satisfiability problem})~\cite{SplintLCLint}.
Ohjelmaan on määritelty jokaista C-kirjaston puskurinkäsittelyfunktiota kohti joukko
\emph{alkuehtoja}, joiden täytyy päteä funktiota kutsuttaessa.
Esimerkiksi kopioitaessa merkkijonoa \texttt{strcpy}-funktiolla ei osoitin kohde- tai lähdepuskuriin
saa olla \texttt{NULL} sekä kohdepuskurin koko täytyy olla vähintään yhtä suuri kuin lähdepuskurin.
Muistia varaavat funktiot ja konstruktiot sen sijaan tuottavat \emph{jälkiehtoja} -
esimerkiksi onnistuneen \texttt{malloc}-funktiokutsun tuloksena on puskuri, jonka
koko annettiin parametrinä.
Näiden symbolisten ehtojen perusteella Splint yrittää selvittää syntyykö
puskurin turvallisen käytön edellyttämistä ehdoista ja
puskurin todellisen koon aiheuttamista ehdoista ristiriita.
Tällöin puskurin ylivuoto on mahdollinen ja Splint antaa siitä varoituksen~\cite{SplintLCLint}.

\section{Fuzzaus}
\label{Fuzzaus}

\emph{Fuzzaus} on yleisesti käytetty keino automaattiseen testisyötteiden generointiin.
Sen ideana on raa'asti luoda suuria määriä mahdollisia syötteitä satunnaislukujen pohjalta siinä toivossa,
että ohjelma ei käsittele niitä oikein~\cite{UnixReliability}.
Muun muassa aiemmin mainittuja muistivirheitä löytyy usein fuzzauksella.
% Tyhmä esimerkki, viimeistele?
% Esimerkiksi useimmat maailmalla olevat äänitiedostot sisältävät suurella todennäköisyydellä vain kaksi äänikanavaa.
% Luultavasti tätä yleisintä tapausta on myöskin testattu enemmän ja muut kanavamäärät ovat jääneet vähemmälle testaamiselle.
% Fuzzeri sen sijaan saattaakin luoda äänitiedostoja, joissa on esimerkiksi 0, -89 tai 4000000 äänikanavaa.
Samaten erikoisia merkkejä sisältävät merkkijonot sekä erittäin suuret tai negatiiviset lukuarvot
voivat olla ongelmallisia~\cite{ViolatingAssumptions}.
Liian suurista luvuista voi aiheutua aritmeettisiä ylivuotoja laskutoimituksissa.
Esimerkiksi monissa tiedostoformaateissa täytyy lukea levyltä tietoa taulukoihin.
Jos esimerkiksi pitää luoda taulukko 128-tavuisia merkkijonoja varten siten,
että taulukon alkioiden lukumäärä luetaan 32-bittisestä kentästä tiedostossa,
voi käydä huonosti muistia varatessa.
Tiedostossa annettu taulukon koko saattaa olla niin suuri,
että laskutoimitus \texttt{128 * koko}, eli taulukon koko tavuina,
saattaakin ylivuotaa 32-bittisen muuttujan siten,
että ohjelma varaakin vain $128$ tavua muistia taulukolle,
kun oikeasti pitäisi varata $2^{32} + 128$ tavua.
Tämä johtaa hyvin nopeasti puskurin ylivuotoon.

Fuzzaukselle hyvin sopivia kohteita ovat muun muassa erinäisten tiedostoformaattien jäsentäjät~\cite{SageArtikkeli,OuluBrowser}:
esimerkiksi web-selaimet joutuvat käsittelemään muun muassa
HTML-, CSS-, PNG- ja JavaScript-muotoisia tiedostoja~\cite{OuluBrowser}.
Tämä on varsin suuri määrä koodia,
joka joutuu käsittelemään syötteenään tiedostoja suoraan lähtökohtaisesti epäluotettavasta lähteestä, Internetistä,
jolloin hyökkäyspinta-alaakin on runsaasti~\cite{OuluBrowser}.

Fuzzaukseen sopivien syötteiden luontiin on olemassa runsaasti tekniikoita
aina yksinkertaisista blackbox-menetelmistä monimutkaisempiin keinoihin:
\begin{itemize}
    \item Yksinkertaisimmillaan syöte voi olla lähes pelkkää satunnaista dataa.
          Esimerkiksi monien Unixin komentorivityökalujen syöte koostuu pelkistä tekstiriveistä
          ilman sen kummempaa rakennetta,
          jolloin jono rivinvaihdoilla eroteltua satunnaisgeneroituja
          tavuja~\cite{UnixReliability} on riittävä syöte ohjelmalle.
    \item Ennalta olemassa olevia kelvollisia syötteitä voidaan \emph{mutatoida} esimerkiksi
          lisäämällä, poistamalla tai muokkaamalla sen osia.
    \item Rakenteellisesti enimmäkseen kelvollisia, mutta normaalisti harvoin esiintyviä syötteitä voidaan
          luoda esimerkiksi kontekstittoman kieliopin tai jonkun muun formaalin syötteen määrittelyn perusteella.
    \item Ohjelman suoritusta voidaan analysoida symbolisesti, jotta voidaan
          selvittää minkälainen vaikutus syötteellä on ohjelman tilaan~\cite{SageArtikkeli}
          ja mahdollisesti karsia pois `hyödyttömiä' syötteitä.
          Esimerkiksi jos voidaan päätellä, että jollekin syötetavulle ei tehdä mitään muuta kuin
          verrata sen yhtäsuuruutta lukuun nolla,
          niin voidaan jättää tarkastelematta kaikki ykköstä suuremmat arvot.
          Tällaisilla arvoillahan ohjelma käyttäytyisi tasan samalla lailla
          kuin jos syötteenä olisi ykkönen,
          jolloin mitään uutta informaatiota ohjelman toiminnasta ei saataisi.

\end{itemize}

Kä\-si\-tel\-lään seuraavaksi joitakin e\-del\-lä\-mai\-nit\-tu\-ja keinoja generoida syötteitä.
%
% \subsection{Testitapausten generointi}
%
Fuzzauksen historia alkaa kirjaimellisesti synkkänä ja myrskyisenä yönä,
jolloin sään aiheuttamat tiedonsiirtovirheet modeemiyhteydessä koituivat ongelmaksi
eräälle alkuperäisen fuzzauspaperin kirjoittajalle.
Tiedonsiirtovirheiden aiheuttamat satunnaiset merkit linjalle aiheutti nimittäin
useiden Unixin peruskomentorivityökalujen kaatumisen~\cite{UnixReliability}.
Tämän innoittamana kehitettiin \emph{fuzz}-niminen työkalu Unix-komentoriviohjelmien testaamiseen.

Fuzz-ohjelma yksinkertaisesti tulostaa standarditulostusvirtaansa satunnaisia tavuja,
jotka yhdistetään Unix-putkella testattavan ohjelman standardisyötevirtaan~\cite{UnixReliability}.
Fuzzin komentoriviparametreillä voi jossain määrin vaikuttaa fuzzin generoimaan syötteeseen:
syötteen pituus on määriteltävissä sekä ASCII-kontrollimerkit saa kytkettyä päälle tai pois.
Kokonaisuudessaan fuzzin generoimassa syötteessä ei ollut lähes ollenkaan rakennetta;
tämä kuitenkin riitti kaatamaan useat Unix-ohjelmat.

Testattavalle ohjelmalle annettu syöte voi ohjelmasta riippuen olla
hyvinkin erityyppistä kuin edellä esitelty tavujonoista koostuva syöte.
Esimerkiksi graafisen käyttöliittymän toteutus toimii usein tapahtumapohjaisesti:
sekä Unixin X11:n, Mac OS X:n Aquan sekä Microsoft Windowsin ikkunanhallinta on toteutettu näin.
Näihin graafisiin järjestelmiin perustuvia ohjelmia voi siten fuzzata lähettämällä
niille esimerkiksi tekaistuja näppäimistön painalluksia tai hiiren liikkeitä vastaavia
sanomia~\cite{X11Fuzz,MacOsFuzz,WinNtFuzz}.

Esimerkki rakenteellisen syötteen generoinnista on \emph{regfuzz}-ohjelma,
joka on kehitelty säännöllisiä lausekkeiden (regular expression) kirjastojen
fuzzaamiseen~\cite{RegFuzz}.
Regfuzz lähtee rakentamaan säännöllistä lauseketta pala kerrallaan sisältä ulospäin:
liikkeelle lähdetään tyhjästä merkkijonosta ja siihen lisätään joko täysin
satunnaisia merkkejä tai valideja rakenteita,
kuten muun muassa sulkeita, merkkijoukkoja (esimerkiksi \texttt{[\^{}a-z]}) ja
operaattoreita (kuten \texttt{+}, \texttt{*}).
Tätä prosessia toistetaan niin kauan kunnes on saatu halutun kokoinen säännöllinen lauseke,
joka voidaan antaa syötteeksi testattavalle säännöllisten lausekkeiden toteutukselle.
Esimerkiksi regfuzzia testattiin alun perin pääosin PCRE-kirjastolle,
joka on käytössä muun muassa useiden selainten Javascript-toteutuksissa sekä Adobe Flashissa.
Regfuzz paljasti PCRE-kirjastosta mahdollisia palvelunestohyökkäyksiä liittyen
virheellisten Unicode-merkkien käsittelyyn sekä tietyntyyppisten validienkin
säännöllisten lausekkeiden aiheuttamaan suureen muistinkulutukseen.

\section{Yhteenveto}
Tässä tutkielmassa tarkasteltiin kahta menetelmää ohjelmistojen tietoturvatestaamiseen,
staattista analyysia ja fuzzausta.
Huomattavaa on, että koska osa tietoturvaongelmista on seurausta ohjelmointivirheistä,
niin tietoturvan parantamiseen soveltuvat ohjelmointivirheitä yleisesti paikantavat ohjelmistotekniikan menetelmät.
Kuitenkin fuzzaus testausmenetelmänä on toimiva menetelmä erityisesti tietoturvaongelmien löytämiseen.


% --- Back matter ---
%
% bibtex is used to generate the bibliography. The babplain style
% will generate numeric references (e.g. [1]) appropriate for theoretical
% computer science. If you need alphanumeric references (e.g [Tur90]), use
%
% \bibliographystyle{babalpha}
%
% instead.

% \bibliographystyle{babplain}
% \bibliography{references-fi}

% \newpage % LAYOUT
\bibliographystyle{babalpha-lf}
\bibliography{viitteet}

\end{document}
