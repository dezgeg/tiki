% --- Template for thesis / report with tktltiki2 class ---

\documentclass[finnish]{tktltiki2/tktltiki2}

% --- General packages ---

\usepackage[utf8]{inputenc}
\usepackage{lmodern}
\usepackage{microtype}
\usepackage{amsfonts,amsmath,amssymb,amsthm,booktabs,color,enumitem,graphicx}
\usepackage[pdftex,hidelinks]{hyperref}

% Automatically set the PDF metadata fields
\makeatletter
\AtBeginDocument{\hypersetup{pdftitle = {\@title}, pdfauthor = {\@author}}}
\makeatother

% --- Language-related settings ---
%
% these should be modified according to your language

% babelbib for non-english bibliography using bibtex
\usepackage[fixlanguage]{babelbib}
\selectbiblanguage{finnish}

% add bibliography to the table of contents
\usepackage[nottoc]{tocbibind}
% tocbibind renames the bibliography, use the following to change it back
\settocbibname{Lähteet}

% --- Theorem environment definitions ---

\newtheorem{lau}{Lause}
\newtheorem{lem}[lau]{Lemma}
\newtheorem{kor}[lau]{Korollaari}

\theoremstyle{definition}
\newtheorem{maar}[lau]{Määritelmä}
\newtheorem{ong}{Ongelma}
\newtheorem{alg}[lau]{Algoritmi}
\newtheorem{esim}[lau]{Esimerkki}

\theoremstyle{remark}
\newtheorem*{huom}{Huomautus}


% --- tktltiki2 options ---
%
% The following commands define the information used to generate title and
% abstract pages. The following entries should be always specified:

\title{Ohjelmistojen haavoittuvuustestaus}
\author{Tuomas Tynkkynen}
\date{\today}
\level{Essee}
\abstract{Tiivistelmä.}

% The following can be used to specify keywords and classification of the paper:

\keywords{avainsana 1, avainsana 2, avainsana 3}
\classification{} % classification according to ACM Computing Classification System (http://www.acm.org/about/class/)
                  % This is probably mostly relevant for computer scientists

% If the automatic page number counting is not working as desired in your case,
% uncomment the following to manually set the number of pages displayed in the abstract page:
%
% \numberofpagesinformation{16 sivua + 10 sivua liitteissä}
%
% If you are not a computer scientist, you will want to uncomment the following by hand and specify
% your department, faculty and subject by hand:
%
% \faculty{Matemaattis-luonnontieteellinen}
% \department{Tietojenkäsittelytieteen laitos}
% \subject{Tietojenkäsittelytiede}
%
% If you are not from the University of Helsinki, then you will most likely want to set these also:
%
% \university{Helsingin Yliopisto}
% \universitylong{HELSINGIN YLIOPISTO --- HELSINGFORS UNIVERSITET --- UNIVERSITY OF HELSINKI} % displayed on the top of the abstract page
% \city{Helsinki}
%

\renewcommand{\baselinestretch}{1.618}

\begin{document}

% --- Front matter ---


\maketitle        % title page
% \makeabstract     % abstract page

% \tableofcontents  % table of contents
% \newpage          % clear page after the table of contents


% --- Main matter ---

\section{Haavoittuvaisuustestaus}

Ohjelmistojen haavoittuvuuksilla voi olla ikäviä seurauksia.
Siksi on toivottavaa, että mahdolliset tietoturva-aukot löydetään mielellään jo ennen tuotantoon asentamista.
Haavoittuvuuksia voi löytää muun muassa manuaalisesti auditoimalla ohjelmakoodi.
Tämä kuitenkin on työlästä ja aikaavievää, joten automatisoituja testausmenetelmiä suositaan enemmän.
Perinteisten ohjelmistotekniikan testausmenetelmien lisäksi tietoturvatestaukseen on kehitelty uudenlaisia testausmenetelmiä.

Testauskeinoja luokitellaan karkeasti \emph{white box}- ja \emph{black box}-testeiksi.
Black box-testauksessa ohjelmistoa testataan antamalla sille syötteitä tavanomaiseen tapaan ilman että sen sisäistä toimintaa tarkastellaan.
Tyypillisesti tämänkaltaisessa testauksessa havainnoidaan ohjelmasta ainoastaan selkeätä ei-toivottua käytöstä, kuten kaatumista tai jumiutumista.

White box-testauksessa sen sijaan ohjelmiston sisäistä rakennetta voidaan automatisoiduin keinoin
tarkastella tai muuttaa, tarpeen tullen binääritasolla tai lähdekooditasolla.

\subsection{Staattinen analyysi}

Staattisen analyysin ohjelmistot, kuten esimerkiksi Coverity~\cite{Coverity} yrittävät löytää ohjelmiston lähdekoodista tyypillisiä ohjelmointivirheitä.
Monista tällaisista ongelmista saattaa olla seuraamuksia tietoturvan kannalta.




% --- Back matter ---
%
% bibtex is used to generate the bibliography. The babplain style
% will generate numeric references (e.g. [1]) appropriate for theoretical
% computer science. If you need alphanumeric references (e.g [Tur90]), use
%
% \bibliographystyle{babalpha}
%
% instead.

% \bibliographystyle{babplain}
% \bibliography{references-fi}


\end{document}
