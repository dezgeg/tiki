\section{Haavoittuvaisuustestaus}

\subsection{Johdanto}
Ohjelmistojen haavoittuvuukset ovat verrattaen ikäviä.
Tietoturva-aukoista aiheutunut ylimääräinen työ voi jo itsessään maksaa miljoonia dollareita~\cite{SageArtikkeli},
puhumattakaan mahdollisista henkilötietojen tai yrityssalaisuuksien vuotamisesta.
Siksi on toivottavaa, että mahdolliset tietoturva-aukot löydetään mielellään jo ennen tuotantoon asentamista.

Tietoturvaongelmia voi toki ohjelmoija löytää tutkimalla lähdekoodia,
tai koodin puuttuessa disassembloimalla ohjelman binääriä.
Tämä kuitenkin on työlästä ja aikaavievää, joten automatisoitu ratkaisu on paikallaan.

\section{Automaattiset menetelmät}

Ohjelmistotekniikan menetelmistä tuttu laadunvarmistustekniikka on automaattiset testit~\cite{Somerville}.
Testausta voidaankin soveltaa tietoturvaongelmien välttämiseen tietyin erotuksin:
sen sijaan, että testataan toivotun toiminnallisuuden olemassaoloa,
testataankin \emph{epätoivotun käytöksen} puutetta~\cite{OuluBrowser}.
Selkeitä epätoivottuja tapahtumia ovat kaatumiset (esimerkiksi luku/kirjoitus muistialueen ulkopuolelta) ja jumiutumiset (esimerkiksi ikisilmukat).
Näissä molemmissa on palvelunestohyökkäyksen (DoS, Denial of Service) uhka,
sekä edeltävässä mahdollisesti myös ulkopuolisen koodin suorituksen mahdollisuus.

Testauskeinoja luokitellaan karkeasti \emph{white box}- ja \emph{black box}-testeiksi.
Black box-testauksessa ohjelmistoa testataan antamalla sille syötteitä tavanomaiseen tapaan ilman että sen sisäistä toimintaa tarkastellaan.
Tyypillisesti tämänkaltaisessa testauksessa havainnoidaan ohjelmasta ainoastaan selkeätä ei-toivottua käytöstä, kuten kaatumista tai jumiutumista.

White box-testauksessa sen sijaan ohjelmiston sisäistä rakennetta voidaan automatisoiduin keinoin
tarkastella tai muuttaa, tarpeen tullen binääritasolla tai lähdekooditasolla.

\subsection{Staattinen analyysi}

Staattisen analyysin ohjelmistot, kuten esimerkiksi Coverity~\cite{Coverity} yrittävät löytää ohjelmiston lähdekoodista tyypillisiä ohjelmointivirheitä.
Monista tällaisista ongelmista saattaa olla seuraamuksia tietoturvan kannalta.


