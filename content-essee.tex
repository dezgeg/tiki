\section{Johdanto}

Ohjelmistojen haavoittuvuukset ovat verrattaen ikäviä.
Tietoturva-aukoista aiheutuneesta ylimääräisestä työstä tietojärjestelmien ylläpitäjille 
sekä menetetystä työajasta voi seurata suuria rahallisia tappioita,
puhumattakaan mahdollisista henkilötietojen tai yrityssalaisuuksien vuotamisesta.
Esimerkiksi Microsoftin IIS-palvelimen haavoittuvuuden avulla levinneestä Code Red-madosta
aiheutui yhteensä noin 2,6 miljardin tappiot~\cite{CodeRed}.
Selvästi on toivottavaa, että ohjelmiston tietoturvasta voidaan varmistua ennen ohjelmiston käyttöönottoa.

Tietoturvaongelmia voi yrittää löytää manuaalisesti tutkimalla ohjelman lähdekoodia,
mikä tietenkin on mahdollista vain ohjelmiston varsinaisille kehittäjille
tai avoimen lähdekoodin ohjelmille.
Lähdekoodin puuttuessa täytyy ensin ohjelmatiedosto takaisinkääntää nk.
\emph{disassembler}-ohjelmalla symboliselle konekielelle,
ja tutkia sitä konekielitasolla.
Tämä kuitenkin on työlästä ja aikaavievää, joten automatisoitu ratkaisu on paikallaan.\fixme

\section{Automaattiset menetelmät}

Ohjelmistotekniikan menetelmistä tuttu laadunvarmistustekniikka on automaattiset testit~\cite{Sommerville}.
Testausta voidaankin soveltaa tietoturvaongelmien välttämiseen tietyin edellytyksin:
sen sijaan, että testataan toivotun toiminnallisuuden olemassaoloa,
testataankin \emph{epätoivotun käytöksen} puutetta~\cite{OuluBrowser}.
Selkeitä epätoivottuja tapahtumia ovat kaatumiset (esimerkiksi luku/kirjoitus muistialueen ulkopuolelta) ja jumiutumiset (esimerkiksi ikisilmukat).
Näissä molemmissa on palvelunestohyökkäyksen (DoS, Denial of Service) uhka,
sekä edeltävässä mahdollisesti myös ulkopuolisen koodin suorituksen mahdollisuus (RCE, Remote Code Execution)~\cite{StaticCodeAnalysis}. \fixme[selitä rce?]

Testauskeinoja voidaan tavallisten funktionaalisten testien tapaan luokitella karkeasti \emph{black box}- ja \emph{white box}-testeiksi \fixme[isolla vai pienellä?] sen mukaan missä määrin
testaaminen kohdistuu ohjelmiston tavanomaisiin rajapintoihin vai\fixme ohjelmiston sisäisen rakenteen toimintaan~\cite{Sommerville}.

Black box-testauksessa \fixme[käännökset?] ohjelmabinääriä ajetaan aivan normaalisti,
ja tutkitaan sen käyttäytymistä ohjelmaan sopivilla syötteillä.
Esimerkiksi palvelinohjelmiston ollessa kyseessä siihen avataan verkkoyhteys,
tai perinteiselle komentoriviohjelmalle komentoriviparametrien kautta.

White box-testauskeinoissa kajotaan ohjelman sisäiseen rakenteeseen.
Tämänkaltaiseen testaukseen on tapoja huomattavasti enemmän.
Esimerkiksi jotkut keinot voivat vaatia pääsyä ohjelman lähdekoodiin,
tai sitten ohjelman suoritusta voidaan analysoida tai muuttaa konekielitasolla.

Tarkastellaan seuraavaksi muutamaa automaattista testausmenetelmää: \emph{staattista analyysiä} ja \emph{fuzzausta}.

\subsection{Staattinen analyysi}

Staattiseen analyysiin perustuvat keinot ovat lähdekoodin analyysiin perustuvia valkoisen laatikon menetelmiä.
Tällaiset menetelmät yrittävät paikantaa ohjelmiston lähdekoodista tyypillisiä ohjelmointivirheitä.
Monista tällaisista ongelmista saattaa olla seuraamuksia tietoturvan kannalta.~\cite{StaticCodeAnalysis}
Yleisin tällainen keino on kääntäjien tarjoamat varoitukset ohjelmiston käännösaikana.
Usein käytettyjä staattisen analyysin työkaluja ohjelmointivirheiden etsintään
ovat nk. \emph{lint}~\cite{Lint}-ohjelmistot eri ohjelmointikielille sekä
esimerkiksi Coverity-yhtiön tuotteet~\cite{Coverity}.

\subsection{Fuzzaus}

\emph{Fuzzaus} on brute force -keino automaattiseen tietoturvatestaukseen.
Fuzzauksen tarkoitus on generoida satunnaisia syötteitä testattavalle ohjelmalle siinä toivossa,
että ohjelma ei käsittele niitä oikein~\cite{UnixReliability}.
Esimerkiksi erikoisia merkkejä sisältävät merkkijonot sekä erittäin suuret tai negatiiviset lukuarvot
voivat olla ongelmallisia~\cite{ViolatingAssumptions}.
Aritmetiikka suurilla luvuilla voi aiheuttaa ylivuotoja laskennassa,
ja taulukoiden varomaton indeksointi voi mennä yli tai ali taulukolle varatun muistialueen.

Fuzzaukselle hyvin sopivia kohteita ovat muun muassa erinäisten tiedostoformaattien jäsentäjät~\cite{SageArtikkeli,OuluBrowser}:
esimerkiksi web-selaimet joutuvat käsittelemään muun muassa
HTML-, CSS-, PNG- ja JavaScript-muotoisia tiedostoja~\cite{OuluBrowser}.
Tämä on varsin suuri määrä koodia,
joka ottaa syötteekseen tiedostoja suoraan lähtökohtaisesti epäturvallisesta Internetistä,
jolloin hyökkäyspinta-alaakin on runsaasti.

Fuzzaukseen sopivien syötteiden luontiin on olemassa runsaasti tekniikoita aina yksinkertaisista black box-menetelmistä monimutkaisempiin keinoihin:
\begin{itemize}
    \item Yksinkertaisimmillaan syöte voi olla pelkkä jono satunnaisgeneroituja tavuja~\cite{UnixReliability}. \fixme[esimerkki]
    \item Ennalta olemassa olevia valideja syötteitä voidaan \emph{mutatoida} lisäämällä, poistamalla, muokkaamalla jne. satunnaisesti.
    \item Rakenteellisesti (enimmäkseen) valideja, mutta normaalisti harvoin esiintyviä syötteitä voidaan luoda esimerkiksi kontekstivapaan kieliopin tai jonkun muun formaalin syötteen määrittelyn perusteella.
    \item Ohjelman suoritusta voidaan analysoida symbolisesti välttämään syötteitä, jolla ei ole vaikutusta ohjelmaan~\cite{SageArtikkeli}.
\end{itemize}

\bibliographystyle{babalpha}
\bibliography{essee}
