\section{Haavoittuvaisuustestaus}

\subsection{Johdanto}
Ohjelmistojen haavoittuvuukset ovat verrattaen ikäviä.
Tietoturva-aukoista aiheutunut ylimääräinen työ voi jo itsessään maksaa miljoonia dollareita~\cite{SageArtikkeli},
puhumattakaan mahdollisista henkilötietojen tai yrityssalaisuuksien vuotamisesta.
Siksi on toivottavaa, että mahdolliset tietoturva-aukot löydetään mielellään jo ennen tuotantoon asentamista.

Tietoturvaongelmia voi toki ohjelmoija löytää tutkimalla lähdekoodia,
tai koodin puuttuessa disassembloimalla ohjelman binääriä.
Tämä kuitenkin on työlästä ja aikaavievää, joten automatisoitu ratkaisu on paikallaan.

\section{Automaattiset menetelmät}

Ohjelmistotekniikan menetelmistä tuttu laadunvarmistustekniikka on automaattiset testit~\cite{Somerville}.
Testausta voidaankin soveltaa tietoturvaongelmien välttämiseen tietyin erotuksin:
sen sijaan, että testataan toivotun toiminnallisuuden olemassaoloa,
testataankin \emph{epätoivotun käytöksen} puutetta~\cite{OuluBrowser}.
Selkeitä epätoivottuja tapahtumia ovat kaatumiset (esimerkiksi luku/kirjoitus muistialueen ulkopuolelta) ja jumiutumiset (esimerkiksi ikisilmukat).
Näissä molemmissa on palvelunestohyökkäyksen (DoS, Denial of Service) uhka,
sekä edeltävässä mahdollisesti myös ulkopuolisen koodin suorituksen mahdollisuus (RCE, Remote Code Execution)~\cite{JokuLähdeTähän}.

Testauskeinoja voidaan tavallisten funktionaalisten testien tapaan luokitella karkeasti \emph{black box}- ja \emph{white box}-testeiksi sen mukaan kuinka paljon
testaaminen kohdistuu ohjelmiston tavanomaisiin rajapintoihin vai ohjelmiston sisäisen rakenteen toimintaan~\cite{Somerville}.

Black box-testauksessa ohjelmabinääriä ajetaan aivan normaalisti, ja tutkitaan sen käyttäytymistä ohjelmaan sopivilla syötteillä.
Esimerkiksi palvelinohjelmiston ollessa kyseessä siihen avataan verkkoyhteys,
tai komentoriviohjelmalle annetaan erinäisiä tiedostoparametrejä.

White box-testauskeinoissa kajotaan ohjelman sisäiseen rakenteeseen.
Tähän tapoja on lukuisia.
Esimerkiksi jotkut keinot voivat vaatia pääsyä ohjelman lähdekoodiin,
tai sitten ohjelman suoritusta voidaan analysoida tai muuttaa konekielitasolla.

\subsection{Staattinen analyysi}

Staattisen analyysin ohjelmistot, kuten esimerkiksi Coverity~\cite{Coverity} yrittävät löytää ohjelmiston lähdekoodista tyypillisiä ohjelmointivirheitä.
Monista tällaisista ongelmista saattaa olla seuraamuksia tietoturvan kannalta.


