\section{Johdanto}

Ohjelmistojen haavoittuvuukset ovat verrattaen ikäviä.
Tietoturva-aukoista aiheutunut ylimääräinen työ voi jo itsessään maksaa miljoonia dollareita~\cite{SageArtikkeli}, \fixme[joku rahasumma jostain viruksesta]
puhumattakaan mahdollisista henkilötietojen tai yrityssalaisuuksien vuotamisesta.
Siksi on toivottavaa, että mahdolliset tietoturva-aukot löydetään ennen ohjelmiston käyttöönottoa.

Tietoturvaongelmia voi toki ohjelmoija löytää tutkimalla lähdekoodia,
tai koodin puuttuessa disassembloimalla ohjelman binääriä. \fixme
Tämä kuitenkin on työlästä ja aikaavievää, joten automatisoitu ratkaisu on paikallaan.

\section{Automaattiset menetelmät}

Ohjelmistotekniikan menetelmistä tuttu laadunvarmistustekniikka on automaattiset testit~\cite{Sommerville}.
Testausta voidaankin soveltaa tietoturvaongelmien välttämiseen tietyin edellytyksin:
sen sijaan, että testataan toivotun toiminnallisuuden olemassaoloa,
testataankin \emph{epätoivotun käytöksen} puutetta~\cite{OuluBrowser}.
Selkeitä epätoivottuja tapahtumia ovat kaatumiset (esimerkiksi luku/kirjoitus muistialueen ulkopuolelta) ja jumiutumiset (esimerkiksi ikisilmukat).
Näissä molemmissa on palvelunestohyökkäyksen (DoS, Denial of Service) uhka,
sekä edeltävässä mahdollisesti myös ulkopuolisen koodin suorituksen mahdollisuus (RCE, Remote Code Execution)~\cite{JokuLahdeTahan}. \fixme

Testauskeinoja voidaan tavallisten funktionaalisten testien tapaan luokitella karkeasti \emph{black box}- ja \emph{white box}-testeiksi \fixme[isolla vai pienellä?] sen mukaan missä määrin
testaaminen kohdistuu ohjelmiston tavanomaisiin rajapintoihin vai\fixme ohjelmiston sisäisen rakenteen toimintaan~\cite{Sommerville}.

Black box-testauksessa \fixme[käännökset?] ohjelmabinääriä ajetaan aivan normaalisti,
ja tutkitaan sen käyttäytymistä ohjelmaan sopivilla syötteillä.
Esimerkiksi palvelinohjelmiston ollessa kyseessä siihen avataan verkkoyhteys,
tai perinteiselle komentoriviohjelmalle komentoriviparametrien kautta.

White box-testauskeinoissa kajotaan ohjelman sisäiseen rakenteeseen.
Tämänkaltaiseen testaukseen on tapoja huomattavasti enemmän.
Esimerkiksi jotkut keinot voivat vaatia pääsyä ohjelman lähdekoodiin,
tai sitten ohjelman suoritusta voidaan analysoida tai muuttaa konekielitasolla.\fixme{selvennä, johdattele seur. kappaleisiin}

\subsection{Staattinen analyysi}

Staattiseen analyysiin perustuvat keinot,
kuten esimerkiksi Coverity-ohjelmisto~\footnote{\url{http://coverity.com}} tai monien kääntäjien varoitukset yrittävät paikantaa ohjelmiston lähdekoodista tyypillisiä ohjelmointivirheitä.
Monista tällaisista ongelmista saattaa olla seuraamuksia tietoturvan kannalta. \fixme[lähde?]

\subsection{Fuzzaus}

\emph{Fuzzaus} on brute force -keino automaattiseen tietoturvatestaukseen.
Fuzzauksen tarkoitus on generoida satunnaisia syötteitä testattavalle ohjelmalle siinä toivossa,
että ohjelma ei käsittele niitä oikein~\cite{UnixReliability}. \fixme[selvennä miksi tämä on toivottavaa]
Fuzzaukselle hyvin sopivia kohteita ovat muun muassa erinäisten tiedostoformaattien jäsentäjät~\cite{SageArtikkeli,OuluBrowser}:
esimerkiksi web-selaimet joutuvat käsittelemään muun muassa HTML-, CSS-, PNG- ja JavaScript-muotoisia tiedostoja~\cite{OuluBrowser}. Tällainen määrä koodia sisältää suuren määrän hyökkäyspinta-alaa.

Fuzzaukseen sopivien syötteiden luontiin on olemassa runsaasti tekniikoita aina yksinkertaisista black box-menetelmistä monimutkaisempiin keinoihin:
\begin{itemize}
    \item Yksinkertaisimmillaan syöte voi olla pelkkä jono satunnaisgeneroituja tavuja~\cite{UnixReliability}. \fixme[esimerkki]
    \item Ennalta olemassa olevia valideja syötteitä voidaan \emph{mutatoida} lisäämällä, poistamalla, muokkaamalla jne. satunnaisesti.
    \item Rakenteellisesti (enimmäkseen) valideja, mutta normaalisti harvoin esiintyviä syötteitä voidaan luoda esimerkiksi kontekstivapaan kieliopin tai jonkun muun formaalin syötteen määrittelyn perusteella.
    \item Ohjelman suoritusta voidaan analysoida symbolisesti välttämään syötteitä, jolla ei ole vaikutusta ohjelmaan~\cite{SageArtikkeli}.
\end{itemize}

\bibliographystyle{babalpha}
\bibliography{essee}
